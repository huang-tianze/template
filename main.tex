% #region
\documentclass[openany,a4paper]{article}
% 语言支持
\usepackage[utf8]{inputenc}
\usepackage[T1]{fontenc}
\usepackage[english]{babel}
\usepackage{xeCJK}
\usepackage{ctex}

% 版面设置
\usepackage[a4paper,margin=1in]{geometry}
\usepackage{setspace}
\usepackage{fancyhdr}
\usepackage{titlesec}
\usepackage{tocloft}

% 代码高亮(minted 需要 Python 和 Pygments)
\usepackage[draft]{minted}
\setminted{breaklines,linenos,breaksymbolleft={}}
\setminted{frame=lines}
\setminted{tabsize=4}
\setminted{encoding=utf8}
\usemintedstyle{colorful}

% 图表支持
\usepackage{graphicx}
\usepackage{float}
\usepackage{caption}
\usepackage{subcaption}
\usepackage{a4} %a4大小
\usepackage{lmodern} %提供部分字体

\usepackage{amsmath,amsthm,amssymb,graphicx}
\usepackage[bookmarks=true,colorlinks,citecolor=blue,linkcolor=black]{hyperref}

% \pagestyle{fancy}
% \fancyhf{}
% \fancyhead[L]{\leftmark}
% \fancyhead[R]{\thepage}
% #endregion

\begin{document}
\tableofcontents

\section{bits}

\subsection{反转二进制}
\inputminted{cpp}{model/bits/bits.h}

\section{数据结构}

\subsection{trie树}
\inputminted{cpp}{model/dataStruct/trie.h}

\subsection{线段树}
\inputminted{cpp}{model/dataStruct/segmentTree.h}

\section{字符串}

\subsection{字符串Hash}
\inputminted{cpp}{model/string/hash.h}

\subsection{KMP}
\inputminted{cpp}{model/string/kmp.h}

\section{Algorithm}

\subsection{STL内置算法}
\inputminted{cpp}{model/algorithm/STL.h}

\subsection{背包DP}
\inputminted{cpp}{model/algorithm/dp.h}

\section{数论}

\subsection{模p下求幂/乘法逆元}
\inputminted{cpp}{model/NumberTheory/modPow.h}

\subsection{线性筛}
\inputminted{cpp}{model/NumberTheory/linearSieve.h}

\subsection{GCD}
\inputminted{cpp}{model/NumberTheory/exgcd.h}

\subsection{中国剩余定理}
\inputminted{cpp}{model/NumberTheory/CRT.h}

\section{杂项(misc)}

\subsection{FAST\_IO}
\inputminted{cpp}{model/misc/fast_io.h}

\subsection{高精度大整数}
\inputminted{cpp}{model/misc/BigInt.h}


\end{document}

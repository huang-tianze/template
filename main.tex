% #region
\documentclass[openany,a4paper]{book}
% 语言支持
% \usepackage[utf8]{inputenc}
\usepackage[T1]{fontenc}
\usepackage[english]{babel}
\usepackage{xeCJK}
\usepackage{ctex}

% 版面设置
\usepackage[a4paper,margin=1in]{geometry}
\usepackage{setspace}
\usepackage{fancyhdr}
\usepackage{titlesec}
\usepackage{tocloft}

% 代码高亮(minted 需要 Python 和 Pygments)
\usepackage{minted}
\setminted{breaklines,linenos}
\renewcommand{\theFancyVerbLine}{\textbf{\small\arabic{FancyVerbLine}}}
\setminted{frame=lines}
\setminted{tabsize=4}
\setminted{encoding=utf8}
\usemintedstyle{colorful}

% 图表支持
\usepackage{graphicx}
\usepackage{float}
\usepackage{caption}
\usepackage{subcaption}
% \usepackage{a4} %a4大小
\usepackage{lmodern} %提供部分字体

\usepackage{amsmath,amsthm,amssymb,graphicx}
\usepackage[bookmarks=true,colorlinks,citecolor=blue,linkcolor=black]{hyperref}

% #endregion
\title{\Huge XCPC算法板子}
\author{Sleeping Thinkers: Tokur233, huang\_tz, sword-a}
\date{\today}
\begin{document}
\maketitle

\tableofcontents

\chapter{Bits}

\section{反转二进制}
\inputminted{cpp}{Bits/bits.h}

\chapter{数据结构 (Data Structure)}

\section{稀疏表(ST)}
\inputminted{cpp}{DataStructure/ST.h}

\section{trie树}
\inputminted{cpp}{DataStructure/trie.h}

\section{树状数组}
\inputminted{cpp}{DataStructure/binaryIndexedTree.h}

\section{线段树}
需要注意,下标从\verb|1|开始; 在外部使用时,\verb|\build(1,n,1)|建树、\verb|getsum(x,y,1,n,1)|查询、\verb|add(x,y,k,1,n,1)|修改。其中\verb|n|为数组长度,\verb|x, y|表示区间$[x,y]$, \verb|k|表示操作数。在外部当黑盒使用的话,参数\verb|1,n,1|是固定不变的,可以自行加一下缩略的重载。
\inputminted{cpp}{DataStructure/segmentTree.h}
\section{并查集}
\inputminted{cpp}{DataStructure/disjointSet.h}

\chapter{字符串 (String)}

\section{字符串Hash}
\inputminted{cpp}{String/hash.h}

\section{KMP}
\inputminted{cpp}{String/kmp.h}

\chapter{算法 (Algorithm)}

\section{STL内置算法}
\inputminted{cpp}{Algorithm/STL.h}

\section{背包DP}
\inputminted{cpp}{Algorithm/dp.h}

\chapter{数论 (Number Theory)}

\section{模p下求幂/乘法逆元}
\inputminted{cpp}{NumberTheory/modPow.h}

\section{欧拉函数}
% TODO:还得加点东西
\inputminted{cpp}{NumberTheory/EulerFunction.h}

\section{线性筛}
\inputminted{cpp}{NumberTheory/linearSieve.h}

\section{exGCD}
\inputminted{cpp}{NumberTheory/exgcd.h}

\section{中国剩余定理}
\inputminted{cpp}{NumberTheory/CRT.h}

\chapter{图论 (Graph Theory)}
\section{最近公共祖先 (LCA)}
这里有朴素算法和倍增算法两种,\verb|n|为节点数,\verb|root|为根节点值,使用邻接表。倍增的预处理使用\verb|preprocess(root,n)|.
\par 朴素算法$O(n)$
\inputminted{cpp}{GraphTheory/LCA-1.h}
倍增算法,预处理$O(n\log n)$,查询$O(\log n)$
\inputminted{cpp}{GraphTheory/LCA-2.h}


\section{Dijkstra}
用于求解有向\textbf{无负边权}图的单源最短路。即起点\verb|s|到所有点的最短路。其中\verb|vis[i]|表示是否遍历到节点\verb|i|,也即是否存在\verb|s|到\verb|i|的路径; \verb|dis[i]|为求解出的最短距离。坏的情况,时间复杂度为$O(n^2)$
\par 对于有负边权的情况需要使用 Bellman-Ford 或 SPFA 。
\inputminted{cpp}{GraphTheory/Dijkstra.h}

\section{A*}
增加了启发式函数\verb|heuristic(a,b)|。恒为0时退化为Dijkstra,小于等于实际最短路时能保证找到路径最短。
\inputminted{cpp}{GraphTheory/Astar.h}

\section{拓扑排序}
用于处理节点有前置要求的任务,也可以用来判是否有环。也即,有向无环图(DAG)才有拓扑排序。
\inputminted{cpp}{GraphTheory/TopoSort.h}

\section{SPFA}
Bellman-Ford的优化版,也用于求解单源最短路,支持负边权。
\inputminted{cpp}{GraphTheory/SPFA.h}

\chapter{杂项 (Misc)}

\section{FAST\_IO}
\inputminted{cpp}{Misc/fast_io.h}

\section{离散化}
\inputminted{cpp}{Misc/Discretize.h}

\section{高精度大整数}
\inputminted{cpp}{Misc/BigInt.h}

\section{遍历全排列}
需要注意的是,使用对\verb|vector<int>|或\verb|std::string|使用\verb|next_permutation()|,是从当前字典序一直遍历到字典序最大的排列。因此,如果要保证所有排列都被遍历到,需要确保最开始的数据需要是最小的字典序。反之,如果写成了字典序最大的,则在一次执行后变成\verb|while(0)|
\inputminted{cpp}{Misc/next_permutation.h}

\section{GBD Task}
\inputminted{json}{Misc/GDB-Task.json}


\end{document}
